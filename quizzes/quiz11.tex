\documentclass[12pt]{exam}

\newcommand{\ds}{\ensuremath{\displaystyle}}

\usepackage{amsmath,amsfonts, amsthm}
\usepackage{multicol}
\usepackage{multirow}
\usepackage{harpoon}
\renewcommand{\arraystretch}{1.5}

\newcommand{\harpvec}[1]{\overrightharp{\ensuremath{\mathbf{#1}}}}
\newcommand{\vect}[1]{\ensuremath{\mathbf{#1}}}
\newcommand{\<}{(}
\renewcommand{\>}{)}
\newcommand{\p}{\partial}

% ref: http://pgfplots.sourceforge.net/gallery.html
% ref: http://tex.stackexchange.com/a/74575/79754
\usepackage{pgfplots}% This uses tikz
\pgfplotsset{compat=newest}% use newest version
\tikzset{LineStyle/.style={smooth, ultra thick, samples=400}}

% \printanswers

\begin{document}

\begin{center}
\fbox{\fbox{\parbox{5.5in}{\centering
MATH 2242-090 | Spring 2016 | Dr. Clontz | Quiz 11
}}}
\end{center}
\vspace{0.1in}
\makebox[\textwidth]{
  Name:\enspace\hrulefill\hrulefill\hrulefill
}

\vspace{12pt}

\begin{itemize}
  \item Each quiz question is labeled with its worth toward your total quiz
        grade for the semester.
  \item On multiple choice problems, you do not need to show your work. No
        partial credit will be given.
  \item On full response problems, show all of your work and give a
        complete solution. When in doubt, don't skip any steps. Partial
        credit will be given at the discretion of the professor.
  \item This quiz is open notes and open book.
  \item This quiz is due at the end of class. Quizzes submitted over one minute
        late will be penalized by \(50\%\).
\end{itemize}

\newpage

\begin{questions}

\question[10]
Evaluate \(\int_C 3xy^2dx + xydy\) where \(C\) is the counter-clockwise
oriented boundary of the rectangle \([0,2]\times[1,3]\). (Hint: Partial credit
will not be given if you attempt to evaluate this directly; try to use a
technique from Chapter 8.)

\vfill

\question[10]
Evaluate \(\int_C (3+y,4y+x)\cdot d\vect s\) where \(C\) is parameterized by
\(\vect r(t)=(2^t,\sin(\pi t))\) for \(0\leq t\leq 1\). (Hint: Partial credit
will not be given if you attempt to evaluate this directly; try to use a
technique from Chapter 8.)

\vfill

% \question[10]
% Which of these is a parametrization of the line segment joining the
% points \((1,0,3)\) and \((2,-2,5)\)?
% \begin{checkboxes}
% \item \(\vect r(t)=(1+t,-2t,3+2t); 0\leq t\leq 1\)
% \item \(\vect r(t)=(\cos t,2\sin t,3\cos t); 0\leq t\leq 2\pi\)
% \item \(\vect r(t)=(t^2,-2e^t,3+5t); -1\leq t\leq 1\)
% \item \(\vect r(t)=(\cos t,-2\sin t,3\cos t); 0\leq t\leq \pi\)
% \item None of these.
% \end{checkboxes}

% \question[10]
% Prove \(\int_C \sqrt y-x+3~ds=\frac{\sqrt{125}-1}{6}\),
% where \(C\) is parametrized by the vector
% function \(\vect r(t)=(3-t,t^2)\) for \(t\in[0,1]\).

% \newpage

% \question[10]
% Calculate the work done by the force \(\vect F=(y,z+x,-2x)\) around the
% curve parametrized by the vector function
% \(\vect r(t)=(\sin t,2\sin t,\cos t)\) for \(t\in[0,\pi]\).



\end{questions}

\end{document}