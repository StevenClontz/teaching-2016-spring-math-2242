\documentclass[12pt]{exam}

\newcommand{\ds}{\ensuremath{\displaystyle}}

\usepackage{amsmath,amsfonts, amsthm}
\usepackage{multicol}
\usepackage{multirow}
\usepackage{harpoon}
\renewcommand{\arraystretch}{1.5}

\newcommand{\harpvec}[1]{\overrightharp{\ensuremath{\mathbf{#1}}}}
\newcommand{\vect}[1]{\ensuremath{\mathbf{#1}}}
\newcommand{\<}{(}
\renewcommand{\>}{)}
\newcommand{\p}{\partial}

% ref: http://pgfplots.sourceforge.net/gallery.html
% ref: http://tex.stackexchange.com/a/74575/79754
\usepackage{pgfplots}% This uses tikz
\pgfplotsset{compat=newest}% use newest version
\tikzset{LineStyle/.style={smooth, ultra thick, samples=400}}

% \printanswers

\begin{document}

\begin{center}
\fbox{\fbox{\parbox{5.5in}{\centering
MATH 2242-090 | Spring 2016 | Dr. Clontz | Quiz 2
}}}
\end{center}
\vspace{0.1in}
\makebox[\textwidth]{
  Name:\enspace\hrulefill\hrulefill\hrulefill
}

\vspace{12pt}

\begin{itemize}
  \item Each quiz question is labeled with its worth toward your total quiz
        grade for the semester.
  \item On multiple choice problems, you do not need to show your work. No
        partial credit will be given.
  \item On full response problems, show all of your work and give a
        complete solution. When in doubt, don't skip any steps. Partial
        credit will be given at the discretion of the professor.
  \item This quiz is open notes and open book.
  \item This quiz is due at the end of class. Quizzes submitted over one minute
        late will be penalized by \(50\%\).
\end{itemize}

\newpage

\begin{questions}

\question[10]
Compute the partial derivative matrix for
\[
  \vect f(x,y) = \<x+e^y,yx^2\>
.\]

\vfill

% \question[10]
% The partial derivative matrix of the differentiable function
% \[
%   \vect f(x,y,z) = \<x,yz,x+3z\>
% \]
% at the point \((1,2,1)\) is
% \[
%   \vect D\vect f(1,2,1)
%     =
%   \left[\begin{matrix}
%     1 & 0 & 0 \\
%     0 & 1 & 0 \\
%     0 & 2 & 3
%   \end{matrix}\right]
% .\]

% Explain why \(\vect f(1.1,1.1,0.9)\approx \<1.1,2.1,3.9\>\) using
% an appropriate linear approximation.


% \vfill\vfill
\end{questions}

\end{document}