\documentclass[12pt]{exam}

\newcommand{\ds}{\ensuremath{\displaystyle}}

\usepackage{amsmath,amsfonts, amsthm}
\usepackage{multicol}
\usepackage{multirow}
\usepackage{harpoon}
\renewcommand{\arraystretch}{1.5}

\newcommand{\harpvec}[1]{\overrightharp{\ensuremath{\mathbf{#1}}}}
\newcommand{\vect}[1]{\ensuremath{\mathbf{#1}}}
\newcommand{\<}{(}
\renewcommand{\>}{)}
\newcommand{\p}{\partial}

% ref: http://pgfplots.sourceforge.net/gallery.html
% ref: http://tex.stackexchange.com/a/74575/79754
\usepackage{pgfplots}% This uses tikz
\pgfplotsset{compat=newest}% use newest version
\tikzset{LineStyle/.style={smooth, ultra thick, samples=400}}

% \printanswers

\begin{document}

\begin{center}
\fbox{\fbox{\parbox{5.5in}{\centering
MATH 2242-090 | Spring 2016 | Dr. Clontz | Quiz 12 (Take-home)
}}}
\end{center}
\vspace{0.1in}
\makebox[\textwidth]{
  Name:\enspace\hrulefill\hrulefill\hrulefill
}

\vspace{12pt}

\begin{itemize}
  \item Each quiz question is labeled with its worth toward your total quiz
        grade for the semester.
  \item On multiple choice problems, you do not need to show your work. No
        partial credit will be given.
  \item On full response problems, show all of your work and give a
        complete solution. When in doubt, don't skip any steps. Partial
        credit will be given at the discretion of the professor.
  \item This take-home
        quiz is open notes and open book. You may work with others as
        long as you don't plagiarize their answers.
  \item This quiz is due at the beginning of class on Monday, May 2.
        Late submissions will not be accepted.
\end{itemize}

\newpage

\begin{questions}

\question[10]
Which of these is a parametrization of the portion of
the surface \(z=x^2+y^2\) above the unit circle in the \(xy\) plane?
\begin{checkboxes}
\item \(\vect \Phi(u,v)=(u^2,v^2,u+v); 0\leq u,v\leq 1\)
\item \(\vect \Phi(x,y)=(x+y,x+y,z^2); 0\leq x,y\leq 1\)
\item \(
  \vect \Phi(r,\theta)
    =
  (r\cos\theta,r\sin\theta,r^2);
  0\leq r\leq 1,0\leq\theta\leq2\pi
\)
\item \(\vect \Phi(u,v)=2u\vect{i}-2v\vect{j}; 0\leq u,v\leq 1\)
\item None of these.
\end{checkboxes}

\question[10]
Prove that the area of the of the triangle with vertices \((0,0,0)\),
\((1,2,-2)\), and \((0,3,3)\) is \(\frac{9\sqrt 2}{2}\) by using
the formula
\(A=\iint_S1dS=
\iint_D\|\frac{\p\vect\Phi}{\p u}\times\frac{\p\vect\Phi}{\p v}\|dA\)
with the parametrization \(\vect\Phi(u,v)=(u,2u+3v,-2u+3v)\). (Hint:
You need to find the domain \(D\) for this parametrization mapping onto
the surface;
this will give you the bounds for the double integral.)

\newpage

\question[10]
Let \(S\) be the oriented surface with an orientation-preserving parametrization
\(\vect\Phi(u,v)=(u,u+v,v^2)\) for \(0\leq u,v\leq 1\). If
\(\vect F=(y,x,z)\) is the
velocity field of a fluid, then show that the flux of the fluid moving through
\(S\) with respect to its orientation is \(1\); that is, verify that
\(\iint_S\vect F\cdot d\vect S=1\).

% \question[10]
% Evaluate \(\int_C 3xy^2dx + xydy\) where \(C\) is the counter-clockwise
% oriented boundary of the rectangle \([0,2]\times[1,3]\). (Hint: Partial credit
% will not be given if you attempt to evaluate this directly; try to use a
% technique from Chapter 8.)

% \vfill

% \question[10]
% Evaluate \(\int_C (3+y,4y+x)\cdot d\vect s\) where \(C\) is parameterized by
% \(\vect r(t)=(2^t,\sin(\pi t))\) for \(0\leq t\leq 1\). (Hint: Partial credit
% will not be given if you attempt to evaluate this directly; try to use a
% technique from Chapter 8.)

% \vfill

% \question[10]
% Which of these is a parametrization of the line segment joining the
% points \((1,0,3)\) and \((2,-2,5)\)?
% \begin{checkboxes}
% \item \(\vect r(t)=(1+t,-2t,3+2t); 0\leq t\leq 1\)
% \item \(\vect r(t)=(\cos t,2\sin t,3\cos t); 0\leq t\leq 2\pi\)
% \item \(\vect r(t)=(t^2,-2e^t,3+5t); -1\leq t\leq 1\)
% \item \(\vect r(t)=(\cos t,-2\sin t,3\cos t); 0\leq t\leq \pi\)
% \item None of these.
% \end{checkboxes}

% \question[10]
% Prove \(\int_C \sqrt y-x+3~ds=\frac{\sqrt{125}-1}{6}\),
% where \(C\) is parametrized by the vector
% function \(\vect r(t)=(3-t,t^2)\) for \(t\in[0,1]\).

% \newpage

% \question[10]
% Calculate the work done by the force \(\vect F=(y,z+x,-2x)\) around the
% curve parametrized by the vector function
% \(\vect r(t)=(\sin t,2\sin t,\cos t)\) for \(t\in[0,\pi]\).



\end{questions}

\end{document}