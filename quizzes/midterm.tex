\documentclass[12pt]{exam}

\newcommand{\ds}{\ensuremath{\displaystyle}}

\usepackage{amsmath,amsfonts, amsthm}
\usepackage{multicol}
\usepackage{multirow}
\usepackage{harpoon}
\renewcommand{\arraystretch}{1.5}

\newcommand{\harpvec}[1]{\overrightharp{\ensuremath{\mathbf{#1}}}}
\newcommand{\vect}[1]{\ensuremath{\mathbf{#1}}}
\newcommand{\<}{(}
\renewcommand{\>}{)}
\newcommand{\p}{\partial}

% ref: http://pgfplots.sourceforge.net/gallery.html
% ref: http://tex.stackexchange.com/a/74575/79754
\usepackage{pgfplots}% This uses tikz
\pgfplotsset{compat=newest}% use newest version
\tikzset{LineStyle/.style={smooth, ultra thick, samples=400}}

% \printanswers

\begin{document}

\begin{center}
\fbox{\fbox{\parbox{5.5in}{\centering
MATH 2242-090 | Spring 2016 | Dr. Clontz | Midterm
}}}
\end{center}
\vspace{0.1in}
\makebox[\textwidth]{
  Name:\enspace\hrulefill\hrulefill\hrulefill
}

\vspace{12pt}

\begin{itemize}
  \item Each question is labeled with its worth toward the grade for
        this midterm out of \(100\).
  \item \textbf{Choose SEVEN of the eight problems to solve. Mark the
        one you don't want graded by marking it with the word ``SKIP''
        in the upperleft corner of the page.} If you fail to do so,
        your highest score from all eight problems will not be counted.
        Note that this means you can earn up to \(105/100\) for the midterm.
  \item Show all of your work and give a
        complete solution. When in doubt, don't skip any steps. Partial
        credit will be given at the discretion of the professor.
  \item You may use at most three pages (front and back)
        of \(8.5\times11\) inch paper for notes.
  \item You may use a calculator no more powerful than a TI-89 (in particular,
        no cell phones are allowed). None of the questions require the
        use of a calculator.
  \item This midterm is due after 70 minutes. Midterms submitted over one minute
        late will be penalized by \(50\%\).
\end{itemize}

\newpage

\begin{questions}

\question[15]
Recall that \(\det(AB)=(\det A)(\det B)\). Evaluate
\[
  \det\left(
    \left[
    \begin{matrix}
       3 &  -4 \\
       1 &  2
    \end{matrix}
    \right]
    \left[
    \begin{matrix}
       1 &  2 \\
       2 &  5
    \end{matrix}
    \right]
  \right)
.\]

\newpage

\question[15]
Compute the partial derivative matrix for
\[
  \vect f(x,y,z) = \<2x+3y,e^z,\sin(yz)\>
.\]

\newpage

\question[15]
Let \(\vect f(u,v)=\<u^2+v^3,2uv\>\),
\(\vect g(x,y)=\<e^{xy},x+y\>\). It follows that
\[
  \vect D\vect f(u,v)
    =
  \left[\begin{matrix}
    2u & 3v^3 \\
    2v & 2u
  \end{matrix}\right]
  \text{~~and~~}
  \vect D\vect g(x,y)
    =
  \left[\begin{matrix}
    ye^{xy} & xe^{xy} \\
    1 & 1
  \end{matrix}\right]
.\]
Use the above matrices and the chain rule to compute
\(\vect D(\vect f\circ\vect g)(0,0)\).

\newpage

\question[15]
Give an approximate value of \(f(1.1,-2.1)\) given the following
information about \(f\):
\[
  f(1,-2)=2
    \hspace{4em}
  \frac{\p f}{\p x}(1,-2)=0
    \hspace{4em}
  \frac{\p f}{\p y}(1,-2)=-1
\]
\[
  \frac{\p^2 f}{\p x^2}(1,-2)=-2
    \hspace{4em}
  \frac{\p^2 f}{\p y^2}(1,-2)=4
    \hspace{4em}
  \frac{\p^2 f}{\p x\p y}(1,-2)=3
\]

\newpage

\question[15]
Prove that \(\vect c(t)=\<t^2,2,t\>\) is a flow line for
the vector field \(\vect F(x,y,z)=\<yz,x-z^2,\frac{1}{2}y\>\).

\newpage

\question[15]
Evaluate \(\int_{0}^{2}\int_{-1}^{2} 2xy+3x^2 ~dydx\).

\newpage

\question[15]
Evaluate \(\int_{0}^{\sqrt\pi}\int_{y}^{\sqrt\pi} 2x^2\cos(xy) ~dxdy\).

\newpage
\question[15]
Express the volume of the pyramid with vertices at \((0,0,0)\),
\((1,0,0)\), \((0,1,0)\), and \((0,0,1)\) as either a double or
triple integral. (Hint: the sides of the pyramid have
equations \(x=0\), \(y=0\), \(z=0\), and \(x+y+z=1\).)
Do not evaluate the integral.

\end{questions}

\end{document}