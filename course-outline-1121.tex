\documentclass[11pt]{article}

\pdfpagewidth 8.5in
\pdfpageheight 11in

\setlength\topmargin{0in}
\setlength\headheight{0in}
\setlength\headsep{0.4in}
\setlength\textheight{8in}
\setlength\textwidth{6in}
\setlength\oddsidemargin{0in}
\setlength\evensidemargin{0in}
\setlength\parindent{0.25in}
\setlength\parskip{0.1in}

\usepackage{amssymb}
\usepackage{amsfonts}
\usepackage{amsmath}
\usepackage{mathtools}
\usepackage{amsthm}

\usepackage{fancyhdr}

\usepackage{enumerate}

      \theoremstyle{plain}
      \newtheorem{theorem}{Theorem}
      \newtheorem{lemma}[theorem]{Lemma}
      \newtheorem{corollary}[theorem]{Corollary}
      \newtheorem{proposition}[theorem]{Proposition}
      \newtheorem{conjecture}[theorem]{Conjecture}
      \newtheorem{question}[theorem]{Question}

      \theoremstyle{definition}
      \newtheorem{definition}[theorem]{Definition}
      \newtheorem{example}[theorem]{Example}
      \newtheorem{game}[theorem]{Game}

      \theoremstyle{remark}
      \newtheorem{remark}[theorem]{Remark}



\pagestyle{fancy}
\renewcommand{\headrulewidth}{0.5pt}
\renewcommand{\footrulewidth}{0pt}
\lfoot{\small \jobname{} -- Updated on \today}
\chead{\small Dr. Clontz -- Fall 2015}
\rfoot{\thepage}
\cfoot{}

\begin{document}

\noindent\textbf{
  MATH 1121 (Calculus for Engineering Technology) Course Outline
}

\section*{1.3 Rectangular Coordinates}

\begin{itemize}
\item Illustrate the following concepts:
  \begin{itemize}
    \item rectangular coordinate system,
    \item \(x\)-axis,
    \item \(y\)-axis,
    \item origin,
    \item quadrants,
    \item coordinates
  \end{itemize}

\item Examples:
  \begin{itemize}
    \item (Example 1) Plot \(A=(2,1)\) and \(B=(-4,-3)\).
    \item (Example 3) Three verticies of a rectangle are
          \(A=(-3,-2)\), \(B=(4,-2)\), \(C=(4,1)\). What is the
          fourth vertex?
  \end{itemize}

\item\textit{
  Suggested homework: 1-9, 15-16, 21-24
}
\end{itemize}

\section*{1.4 The Graph of a Function}

\begin{itemize}
\item Definition of a function \(y=f(x)\).
\item Overview of the table method for plotting function graphs.
\item Examples:
  \begin{itemize}
  \item (Example 1) Graph \(f(x)=3x-5\).
  \item (Example 3) Graph \(y=3x-5\).
  \item (Example 4) Graph \(y=\sqrt{x+1}\).
  \item (Example 6) Graph
    \[
      f(x)=
        \begin{cases}
          2x+1 & x\leq 1 \\
          6-x^2 & x>1
        \end{cases}
    \]
  \end{itemize}

\item\textit{
  Suggested homework: 1-12, 37-40
}
\end{itemize}

\section*{2.1 Some Basic Definitions}

\begin{itemize}
\item Distance Formula
  \begin{itemize}
    \item \(d=\sqrt{(x_2-x_1)^2+(y_2-y_1)^2}\)
    \item (Example 2) Find the distance between \((3,-1)\) and \((-2,-5)\).
  \end{itemize}
\item Slope Formula
  \begin{itemize}
    \item \(m=\frac{y_2-y_1}{x_2-x_1}\)
    \item (Example 3) Find the slope of the line joining
          \((3,-5)\), \((-2,-6)\).
    \item (Example 4) Find the slope of the line joining
          \((3,4)\), \((4,-6)\).
  \end{itemize}
\item Identify parallel/perpendicular lines by slopes.
  \begin{itemize}
    \item Parallel: \(m_1=m_2\)
    \item Perpendicular: \(m_1=-\frac{1}{m_2}\)
    \item (Example 7) Prove that the triangle with vertices
          \(A=(-5,3)\), \(B=(6,0)\), and \(C=(5,5)\) is a right triangle.
  \end{itemize}
\item\textit{
  Suggested HW: 1-20, 29-36
}
\end{itemize}

\section*{2.2 The Straight Line}

\begin{itemize}
\item Point-slope form
  \begin{itemize}
    \item \(y-y_1=m(x-x_1)\)
    \item (Example 2) Find the equation of the line passing through
          \((2,-1)\) and \((6,2)\).
  \end{itemize}
\item Slope-intercept form
  \begin{itemize}
    \item \(y=mx+b\)
    \item (Example 4) Find the slope and \(y\)-intercept of the straight line
          with equation \(2y+4x-5=0\).
  \end{itemize}
\item\textit{
  Suggested HW: 1-21, 33-40
}
\end{itemize}

\section*{Remaining Topics}

\begin{itemize}
  \item 2.3 The Circle
  \item 2.4 The Parabola
  \item 2.5 The Ellipse
  \item 2.6 The Hyperbola
  \item 2.7 Translation of Axes
  \item 1.2 Algebraic Functions
  \item 3.1 Limits
  \item 3.2 The Slope of a Tangent to a Curve
  \item 3.3 The Derivative
  \item 3.4 The Derivative as an Instantaneous Rate of Change
  \item 3.5 Derivatives of Polynomials
  \item 3.6 Derivatives of Products and Quotients of Functions
  \item 3.7 The Derivative of a Power of a Function
  \item 3.8 Differentiation of Implicit Functions
  \item 3.9 Higher Derivatives
  \item 4.1 Tangents and Normals
  \item 4.4 Related Rates
  \item 4.5 Using Derivatives in Curve Sketching
  \item 4.6 More on Curve Sketching
  \item 4.7 Applied Maximum and Minimum Problems
  \item 4.8 Differentials and Linear Approximations
  \item 5.1 Antiderivatives
  \item 5.2 The Indefinite Integral
  \item 5.3 The Area Under a Curve
  \item 5.4 The Definite Integral
  \item 7.1 The Trigonometric Functions
  \item 7.2 Basic Trigonometric Relations
  \item 7.3 Derivatives of the Sine and Cosine Functions
  \item 7.4 Derivatives of the Other Trigonometric Functions
  \item 8.1 Exponential and Logarithmic Functions
  \item 8.2 Derivative of the Logarithmic Functions
  \item 8.3 Derivative of the Exponentials Function
  \item 9.1 The General Power Formula
  \item 9.2 Basic Logarithmic Form
  \item 9.3 Exponential Form
  \item 9.4 Basic Trigonometric Forms
\end{itemize}

\end{document}