\documentclass[11pt]{article}


\pdfpagewidth 8.5in
\pdfpageheight 11in

\setlength\topmargin{0in}
\setlength\headheight{0in}
\setlength\headsep{0.4in}
\setlength\textheight{8in}
\setlength\textwidth{6in}
\setlength\oddsidemargin{0in}
\setlength\evensidemargin{0in}
\setlength\parindent{0.25in}
\setlength\parskip{0.1in}

\usepackage{amssymb}
\usepackage{amsfonts}
\usepackage{amsmath}
\usepackage{mathtools}
\usepackage{amsthm}

\usepackage{fancyhdr}

\usepackage{enumerate}

      \theoremstyle{plain}
      \newtheorem{theorem}{Theorem}
      \newtheorem{lemma}[theorem]{Lemma}
      \newtheorem{corollary}[theorem]{Corollary}
      \newtheorem{proposition}[theorem]{Proposition}
      \newtheorem{conjecture}[theorem]{Conjecture}
      \newtheorem{question}[theorem]{Question}

      \theoremstyle{definition}
      \newtheorem{definition}[theorem]{Definition}
      \newtheorem{example}[theorem]{Example}
      \newtheorem{game}[theorem]{Game}

      \theoremstyle{remark}
      \newtheorem{remark}[theorem]{Remark}



\pagestyle{fancy}
\renewcommand{\headrulewidth}{0.5pt}
\renewcommand{\footrulewidth}{0pt}
\lfoot{\small \jobname{} -- Updated on \today}
\chead{\small Dr. Clontz -- Spring 2016}
\rfoot{\thepage}
\cfoot{}

\newcommand{\vect}[1]{\mathbf{#1}}
\newcommand{\veci}{\vect i}
\newcommand{\vecj}{\vect j}
\newcommand{\veck}{\vect k}
\newcommand{\<}{\left(}
\renewcommand{\>}{\right)}
\newcommand{\Arctan}{\textrm{Arctan}}
\newcommand{\p}{\partial}
\newcommand{\mb}{\mathbb}
\newcommand{\sgn}{\textrm{sgn}}
\renewcommand{\div}{\textrm{div}\,}
\newcommand{\curl}{\textrm{curl}\,}
\newcommand{\scurl}{\textrm{scurl}\,}
\newcommand{\dvar}{\,d}

\renewcommand{\labelitemii}{\tiny$\blacksquare$}

\begin{document}

\noindent\textbf{
  MATH 2242 (Calculus IV) Course Outline
}
| Vector Calculus (Marsden)

% \section*{1.2 The Inner Product, Length, and Distance}

% \begin{itemize}
%   \item Inner/Dot Product
%     \begin{itemize}
%       \item \(\vect{a}\cdot\vect{b}=a_1b_1+a_2b_2+a_3b_3\)
%     \end{itemize}
%   \item Norm/Magnitude/Length
%     \begin{itemize}
%       \item \(\|\vect{a}\|=\sqrt{\vect{a}\cdot\vect{a}}\)
%       \item Alternate dot product:
%             \(\vect{a}\cdot\vect{b}=\|\vect{a}\|\|\vect{b}\|\cos\theta\)
%     \end{itemize}
%   \item Normalization/Direction
%     \begin{itemize}
%       \item \(\frac{\vect{a}}{\|\vect{a}\|}\)
%     \end{itemize}
%   \item Distance
%     \begin{itemize}
%       \item \(\|\vect{b}-\vect{a}\|\)
%     \end{itemize}
%   \item Inequalities
%     \begin{itemize}
%       \item \(|\vect{a}\cdot\vect{b}|\leq\|\vect{a}\|\|\vect{b}\|\)
%       \item \(\|\vect{a}+\vect{b}\|\leq\|\vect{a}\|+\|\vect{b}\|\)
%     \end{itemize}
% \end{itemize}

% \section*{1.3 Matricies, Determinants, and the Cross Product}

% \begin{itemize}
%   \item Matrices
%     \begin{itemize}
%       \item
%         \(
%           \begin{bmatrix}
%             x_{11} & x_{12} \\
%             x_{21} & x_{22}
%           \end{bmatrix}
%         \)
%       \item
%         \(
%           \begin{bmatrix}
%             x_{11} & x_{12} & x_{13} \\
%             x_{21} & x_{22} & x_{23} \\
%             x_{31} & x_{32} & x_{33}
%           \end{bmatrix}
%         \)
%     \end{itemize}
%   \item Determinants
%     \begin{itemize}
%       \item
%         \(
%           \det\left(\begin{bmatrix}
%             x_{11} & x_{12} \\
%             x_{21} & x_{22}
%           \end{bmatrix}\right)
%             =
%           x_{11}x_{22} - x_{12}x_{21}
%         \)
%       \item
%         \(
%           \det\left(\begin{bmatrix}
%             x_{11} & x_{12} & x_{13} \\
%             x_{21} & x_{22} & x_{23} \\
%             x_{31} & x_{32} & x_{33}
%           \end{bmatrix}\right)
%         \)

%         \(
%             =
%           x_{11}\det\left(\begin{bmatrix}
%             x_{22} & x_{23} \\
%             x_{32} & x_{33}
%           \end{bmatrix}\right) -
%           x_{12}\det\left(\begin{bmatrix}
%             x_{21} & x_{23} \\
%             x_{31} & x_{33}
%           \end{bmatrix}\right) +
%           x_{13}\det\left(\begin{bmatrix}
%             x_{21} & x_{22} \\
%             x_{31} & x_{32}
%           \end{bmatrix}\right)
%         \)
%       \item
%         \(\displaystyle
%           \det(A) = \sum_{i=1}^{n}(-1)^{i+1}x_{1i}\det(A_i)
%         \)
%     \end{itemize}
%   \item Cross-Product
%     \begin{itemize}
%       \item
%         \(
%           \<a_1,a_2,a_3\>\times\<b_1,b_2,b_3\>
%             =
%           \det\left(\begin{bmatrix}
%             \veci & \vecj & \veck \\
%             a_{1} & a_{2} & a_{3} \\
%             b_{1} & b_{2} & b_{3}
%           \end{bmatrix}\right)
%         \)
%       \item
%         \(
%           \|\vect{a}\times\vect{b}\|
%             =
%           \|\vect{a}\|\|\vect{b}\|\sin\theta
%         \)
%       \item
%         \(\vect{a}\), \(\vect{b}\), \(\vect{a}\times\vect{b}\) are mutually
%         orthogonal and follow the right-hand-rule
%     \end{itemize}
%   \item Triple Scalar Product
%     \begin{itemize}
%       \item
%         \(
%           (\vect{a}\times\vect{b})\cdot\vect{c}
%             =
%           \det\left(\begin{bmatrix}
%             \vect{a} \\
%             \vect{b} \\
%             \vect{c}
%           \end{bmatrix}\right)
%         \)
%     \end{itemize}
%   \item Plane Equation
%     \begin{itemize}
%       \item \(\vect{n}\cdot(\vect{x}-\vect{P})=0\)
%       \item \(n_1(x-P_1)+n_2(y-P_2)+n_3(z-P_3)=0\)
%     \end{itemize}
% \end{itemize}

\section*{1.5 \(n\)-Dimensional Euclidean Space}

\begin{itemize}
  \item \(\mb R\), \(\mb R^2\), \(\mb R^3\), \(\mb R^n\)
  \item Addition
    \begin{itemize}
      \item
        \(
          \<x_1,x_2,\dots,x_n\> + \<y_1,y_2,\dots,y_n\>
            =
          \<x_1+y_1,x_2+y_2,\dots,x_n+y_n\>
        \)
    \end{itemize}
  \item Scalar multiplication
    \begin{itemize}
      \item
        \(
          \alpha\<x_1,x_2,\dots,x_n\>
            =
          \<\alpha x_1,\alpha x_2,\dots,\alpha x_n\>
        \)
    \end{itemize}
  \item Inner/Dot Product
    \begin{itemize}
      \item
        \(
          \<x_1,x_2,\dots,x_n\>\cdot\<y_1,y_2,\dots,y_n\>
            =
          \sum_{i=1}^n x_iy_i
        \)
    \end{itemize}
  \item Norm/Length/Magnitude
    \begin{itemize}
      \item
        \(
          \|\vect x\| = (\vect x\cdot\vect x)^{1/2}
        \)
    \end{itemize}
  \item Standard basis vectors
    \begin{itemize}
      \item
        \(
          \vect{e}_1=\<1,0,\dots,0\>
        \),
        \(
          \vect{e}_2=\<0,1,\dots,0\>
        \), \dots,
        \(
          \vect{e}_n=\<0,0,\dots,1\>
        \)
    \end{itemize}
  \item Theorems
    \begin{itemize}
      \item
        \(
          (\alpha\vect{x}+\beta\vect{y})\cdot\vect{z}
            =
          \alpha(\vect{x}\cdot\vect{z}) + \beta(\vect{y}\cdot\vect{z})
        \)
      \item Prove the above theorem.
      \item
        \(
          \vect{x}\cdot\vect{y}
            =
          \vect{y}\cdot\vect{x}
        \)
      \item
        \(
          \vect{x}\cdot\vect{x} \geq 0
        \)
      \item
        \(
          \vect{x}\cdot\vect{x} = 0
        \)
        if and only if
        \(
          \vect{x}=\vect{0}
        \)
      \item
        \(
          |\vect{x}\cdot\vect{y}|
            \leq
          \|\vect{x}\|\|\vect{y}\|
        \)
        (the Cauchy-Schwarz inequality)
      \item (Example) Prove the Cauchy-Schwarz inequality.
      \item
        \(
          \|\vect{x}+\vect{y}\|
            \leq
          \|\vect{x}\|+\|\vect{y}\|
        \)
        (the triangle inequality)
      \item (Example) Prove the triangle inequality.
    \end{itemize}
  \item Matrices
    \begin{itemize}
      \item
        \(
          A
            =
          \begin{bmatrix}
            a_{11} & a_{12} & \dots  & a_{1n} \\
            a_{21} & a_{22} & \dots  & a_{2n} \\
            \vdots & \vdots & \ddots & \vdots \\
            a_{m1} & a_{m2} & \dots  & a_{mn}
          \end{bmatrix}
        \)
      \item Addition \(A+B\)
      \item Scalar Mutiplication \(\alpha A\)
      \item Transposition \(A^T\)
    \end{itemize}
  \item Vectors as Matrices
    \begin{itemize}
      \item
        \(
          \vect{a}=\<a_1,a_2,\dots,a_n\>
            =
          \begin{bmatrix}
            a_{1}  \\
            a_{2}  \\
            \vdots \\
            a_{n}
          \end{bmatrix}
        \)
      \item
        \(
          \vect{a}^T
            =
          \begin{bmatrix}
            a_{1} & a_{2} & \cdots & a_{n}
          \end{bmatrix}
        \)
    \end{itemize}
  \item Matrix Multiplication
    \begin{itemize}
      \item
        If \(A\) has \(m\) rows and \(B\) has \(n\) columns,
        then \(M=AB\) is an \(m\times n\) matrix.
      \item
        Coordinate \(ij\) of \(M=AB\) is given by
        \(m_{ij}=\vect{a_i}\cdot\vect{b_j}\)
        where \(\vect{a_i}^T\) is the \(i\)th row of \(A\)
        and \(\vect{b_j}\) is the \(j\)th column of \(B\).
      \item
        (Example 4) Compute \(AB\) and \(BA\) for
        \[
          A =
          \begin{bmatrix}
            1 & 0 & 3 \\
            2 & 1 & 0 \\
            1 & 0 & 0
          \end{bmatrix}
        \]
        \[
          B =
          \begin{bmatrix}
            0 & 1 & 0 \\
            1 & 0 & 0 \\
            0 & 1 & 1
          \end{bmatrix}
        \]
      \item
        (Example 5) Compute \(AB\) for
        \[
          A =
          \begin{bmatrix}
            2 & 0 & 1 \\
            1 & 1 & 2
          \end{bmatrix}
        \]
        \[
          B =
          \begin{bmatrix}
            1 & 0 & 2 \\
            0 & 2 & 1 \\
            1 & 1 & 1
          \end{bmatrix}
        \]
    \end{itemize}
  \item Matrices as Linear Transformations
    \begin{itemize}
      \item An \(m\times n\) matrix \(A\) gives a function from \(\mb R^n\)
            to \(\mb R^m\): \(\vect x \mapsto A\vect x\)
      \item This linear transformation satsifies
            \(
              A(\alpha\vect x + \beta\vect y)
                =
              \alpha A\vect x + \beta A\vect y
            \)
      \item (Example 7) Express \(A\vect x\) where \(x=\<x_1,x_2,x_3\>\) and
        \(
          A =
          \begin{bmatrix}
             1 &  0 &  3 \\
            -1 &  0 &  1 \\
             2 &  1 &  2 \\
            -1 &  2 &  2
          \end{bmatrix}
        \).
      \item (Example) Compute where the points
          \((-1,-1,0)\),
          \((0,1,0)\),
          \((1,-1,1)\), and
          \((2,1,1)\)
        in \(\mb R^3\) get
        mapped to in \(\mb R^4\) by \(A\vect x\) from the previous example.
        Then plot the projections of the original points in \(\mb R^3\) onto
        their first two coordinates in \(\mb R^2\),
        and compare this with the projection plot
        of their images in \(\mb R^4\) onto their first two coordinates in
        \(\mb R^2\).
    \end{itemize}
  \item Identity and Inverse
    \begin{itemize}
      \item
        The \(n\times n\) identity matrix \(I\) satisfies \(i_{jj}=1\)
        and \(i_{jk}=0\) when \(j\not=k\). That is:
        \(
          I =
          \begin{bmatrix}
            1      & 0      & \cdots & 0      \\
            0      & 1      & \cdots & 0      \\
            \vdots & \vdots & \ddots & \vdots \\
            0      & 0      & \cdots & 1
          \end{bmatrix}
        \)
      \item
        If \(AA^{-1}=A^{-1}A=I\), then \(A\) is invertable and
        \(A^{-1}\) is its inverse.
    \end{itemize}
  \item Determinant
    \begin{itemize}
      \item Let \(A_i\) be the submatrix of \(A\) with the first column
      and \(i\)th row removed. Then
        \(
          \det(A)
            =
          \sum_{i=1}^n
          (-1)^{i+1}
          a_{1i}\det(A_i)
        \)
      \item (Example) Prove that
        \[
          \det
          \begin{bmatrix}
            a_1 & a_2 \\
            b_1 & b_2
          \end{bmatrix}
            =
          a_1b_2-a_2b_1
        \]
      and
        \[
          \det
          \begin{bmatrix}
            a_1 & a_2 & a_3 \\
            b_1 & b_2 & b_3 \\
            c_1 & c_2 & c_3
          \end{bmatrix}
            =
          a_1\det
          \begin{bmatrix}
            b_2 & b_3 \\
            c_2 & c_3
          \end{bmatrix}
            -
          a_2\det
          \begin{bmatrix}
            b_1 & b_3 \\
            c_1 & c_3
          \end{bmatrix}
            +
          a_3\det
          \begin{bmatrix}
            b_1 & b_2 \\
            c_1 & c_2
          \end{bmatrix}
        \]
        \[
            =
          (a_1b_2c_3+a_2b_3c_1+a_3b_1c_2)-(a_1b_3c_2+a_2b_1c_3+a_3b_2c_1)
        \]
      \item
        (Example) Prove that the inverse of the matrix
        \(
          A = \begin{bmatrix}
            a_1 & a_2 \\
            b_1 & b_2
          \end{bmatrix}
        \) is
        \(
          \frac{1}{\det A}
          \begin{bmatrix}
            b_2 & -a_2 \\
            -b_1 & a_1
          \end{bmatrix}
        \).
      \item An \(n\times n\)
        matrix is invertable if and only if its determinant is nonzero.
    \end{itemize}
  \item\textit{
    HW: 1-18, 21-24
  }
\end{itemize}

  % \item 2.1 The Geometry of Real-Valued Functions

\section*{2.3 Differentiation}

\begin{itemize}
  \item Functions \(\mb R^n\to\mb R^m\)
    \begin{itemize}
      \item \(\vect{f}:\mb R^n\to\mb R^m\)
      \item
        \(
          \vect{f}(\vect{x})
            =
          \<f_1(\vect{x}),\dots,f_m(\vect{x})\>
        \) where \(f_i:\mb R^n\to \mb R\)
    \end{itemize}
  \item Partial Derivative Matrix %TODO: change \vect D to just D since it's a matrix
    \begin{itemize}
      \item
        \(\displaystyle
          \vect{D}\vect{f}(\vect{x})
            =
          \begin{bmatrix}
            \frac{\p f_1}{\p x_1}(\vect x) &
            \cdots &
            \frac{\p f_1}{\p x_n}(\vect x)
            \\
            \vdots & \ddots & \vdots
            \\
            \frac{\p f_m}{\p x_1}(\vect x) &
            \cdots &
            \frac{\p f_m}{\p x_n}(\vect x)
          \end{bmatrix}
        \)
      \item
        We say \(\vect f\) is differentiable at \(\vect{x}_0\) if
        \(
          \vect{f}(\vect x_0+\vect h)
            \approx
          \vect{f}(\vect x_0)+[\vect{D}\vect{f}(\vect{x}_0)]\vect{h}
        \)
        whenever \(\vect h \approx \vect 0\).
      \item (Example) Prove that this is equivalent to saying
        \(
          \vect{f}(\vect x)
            \approx
          \vect{f}(\vect x_0)+[\vect{D}\vect{f}(\vect{x}_0)](\vect x-\vect x_0)
        \)
        whenever \(\vect x\approx\vect x_0\).
      \item (Example) Let \(\vect f:\mb R^2\to\mb R^2\) be defined by
        \(\vect f(x,y)=\<x^2+y^2,xy\>\), and let
        \(\vect{T}=\vect{D}\vect{f}(1,0)\). Compute
        \(\vect{f}(1.1,-0.1)\) and \(\vect{f}(1,0)+\vect{T}\<0.1,-0.1\>\).
      \item If each \(\frac{\p f_i}{\p x_j}:\mb R^n\to\mb R\)
        is a continuous function near \(\vect x_0\), then we say \(\vect{f}\)
        is strongly differentiable or class \(C^1\) at \(\vect x_0\).
        All \(C^1\) functions are differentiable.
    \end{itemize}
  \item Gradient
    \begin{itemize}
      \item If \(f:\mb R^n\to\mb R\), then the gradient vector function
        \(\nabla f:\mb R^n\to\mb R^n\) is defined by
        \(
          \nabla f(\vect x)
            =
          (\vect D f(\vect x))^T
            =
          \<\frac{\p f}{\p x_1}(\vect x),\dots,\frac{\p f}{\p x_n}(\vect x)\>
        \)
      \item
        \(
          [\vect D f(\vect x)]\vect h
            =
          \nabla f(\vect x) \cdot \vect h
        \)
    \end{itemize}
  \item Linearizations and Tangent Hyperplanes
    \begin{itemize}
      \item For \(\vect f:\mb R^n\to\mb R^m\) and a point \(\vect x_0\in\mb R^n\),
        let the linearization of \(\vect f\) at \(\vect x_0\) be
          \(
            \vect L(\vect x)
              =
            \vect{f}(\vect x_0)+[\vect{D}\vect{f}(\vect{x_0})](\vect{x}-\vect{x}_0)
          \).
        Note \(\vect f(\vect x)\approx\vect L(\vect x)\) whenever
        \(\vect x\approx\vect x_0\).
      \item (Example 5) Recall that the tangent plane to a surface \(z=f(x,y)\)
        given by \(f:\mb R^2\to\mb R\) passing through \(\vect x_0\in\mb R^3\)
        is given by the normal vector \(\nabla f\).
        Show that \(z=L(x,y)\) gives an equation for the tangent plane to the
        surface \(z=x^2+y^4+e^{xy}\) at the point \((1,0,2)\).
    \end{itemize}
  \item\textit{
    HW: 1-3, 5-21
  }
\end{itemize}

  % \item 2.4 Introduction to Paths and Curves

\section*{2.5 Properties of the Derivative}

\begin{itemize}
  \item Sum/Product/Quotient Rules
    \begin{itemize}
      \item \(\vect D[\alpha \vect f]=\alpha\vect D\vect f\)
      \item \(\vect D[\vect f+\vect g]=\vect D\vect f+\vect D\vect g\)
      \item
        \(
          \vect D[fg]=g\vect Df + f\vect Dg
        \)
      \item
        \(
          \vect D[\frac{f}{g}]
            =
          \frac{
            g\vect Df - f\vect Dg
          }{
            g^2
          }
        \)
      \item (Example) Prove the sum rule above.
    \end{itemize}
  \item Chain Rule
    \begin{itemize}
      \item
        \(
          \vect D [\vect f\circ \vect g]
            =
          \vect D\vect f(\vect g)\vect D\vect g
        \)
      \item
        (Example) Find the rate of change of \(f(x,y)=x^2+y^2\) along
        the path \(\vect c(t)=\<t^2,t\>\) when \(t=1\).
      \item
        (Example 2) Verify the Chain Rule for \(f(u,v,w)=u^2+v^2-w\)
        and \(\vect g(x,y,z)=\<x^2y,y^2,e^{-xz}\>\).
      \item
        (Example 3) Compute \(\vect D[\vect f\circ\vect g](1,1)\) where
        \(\vect f(u,v)=\<u+v,u,v^2\>\) and \(\vect g(x,y)=\<x^2+1,y^2\>\).
    \end{itemize}
  \item\textit{
    HW: 6-13, 15-16
  }
\end{itemize}

  % \item 2.6 Gradients and Directional Derivatives

\section*{3.2 Taylor's Theorem}

\begin{itemize}
  \item Single-variable Taylor Series
    \begin{itemize}
      \item
        \(\displaystyle
          f(x)
            =
          \sum_{n=0}^\infty \frac{f^{(n)}(x_0)}{n!}(x-x_0)^n
        \)

        \(\displaystyle
            =
          f(x_0)+f'(x_0)(x-x_0)+\frac{1}{2}f'(x_0)(x-x_0)^2
          +\frac{1}{6}f'(x_0)(x-x_0)^3+\dots
        \)
      \item
        \(\displaystyle
          f(x)
            \approx
          \sum_{n=0}^m \frac{f^{(n)}(x_0)}{n!}(x-x_0)
        \)
    \end{itemize}
  \item First-Order Taylor Formula
    \begin{itemize}
      \item
        \(
          f(\vect x)
            \approx
          L(\vect x)
            =
          f(\vect x_0) + [\vect Df(\vect x_0)](\vect x - \vect x_0)
            =
          f(\vect x_0)
            +
          \sum_{i=1}^{n}\frac{\p f}{\p x_i}(\vect x_0)(x_i - x_{0i})
        \)
    \end{itemize}
  \item Second-Order Taylor Formula
    \begin{itemize}
      \item
        \(
          f(\vect x)
            \approx
          f(\vect x_0)
            +
          \sum_{i=1}^{n}\frac{\p f}{\p x_i}(\vect x_0)(x_i-x_{0,i})
            +
          \frac{1}{2}
          \sum_{i,j=1}^{n}\frac{\p^2 f}{\p x_i\p x_j}(\vect x_0)(x_i-x_{0,i})(x_j-x_{0,j})
        \)
      \item
        (Example) Use the second-order Taylor formula for
        \(f(x,y)=\sqrt{x+2y}\) near the point \((2,1)\) to approximate
        \(\sqrt{4.05}\).
      \item
        (Example 3) Find linear and quadratic functions of \(x,y\) which
        approximate \(f(x,y)=\sin(xy)\) near the point \((1,\pi/2)\).
    \end{itemize}
  \item\textit{
    HW: 3-7, 12
  }
\end{itemize}

  % \item 4.1 Acceleration and Newton's Second Law
  % \item 4.2 Arc Length

\section*{4.3 Vector Fields}

\begin{itemize}
  \item Vector Fields
    \begin{itemize}
      \item A vector field is a map \(f:\mb R^n\to\mb R^n\) assinging an
        \(n\)-dimensional vector to each point in \(\mb R^n\)
      \item (Example 1) The velocity field of a fluid may be modeled as a
        vector field.
      \item (Example 2) Sketch the rotary motion given by the vector
        field \(\vect V(x,y)=\<-y,x\>\).
    \end{itemize}
  \item Gradient Vector Fields
    \begin{itemize}
      \item \(\nabla f = \<\frac{\p f}{\p x_1},\dots,\frac{\p f}{\p x_n}\>\)
      \item (Example) The derivative of a scalar function
        \(f:\mb R^n\to \mb R\) in the direction given by a unit vector
        \(\vect v\)
        is given by \(\nabla f\cdot \vect v\). Show that the maximum value
        of a directional derivative for a fixed point
        is given by \(\|\nabla f\|\)
        and attained by the direction \(\frac{1}{\|\nabla f\|}\nabla f\).
      \item (Example 4) If temperature is given by \(T(x,y,z)\), then the
        energy or heat flux field is given by \(\vect J = -k\nabla T\) where
        \(k\) is the conductivity of the body. Level sets are called isotherms.
      \item (Example 5) The gravitational potential of bodies with mass \(m,M\)
        is given by \(V=-\frac{mMG}{r}\) where \(G\) is the gravitational
        constant and \(r\) is the distance between the bodies, and the
        gravitational force field is given by \(\vect F=-\nabla V\).
        Show that \(\vect F = -\frac{mMG}{r^3}\vect r\), where \(\vect r\)
        is the vector pointing from the center of mass \(M\) to the center
        of mass \(m\).
      \item A vector field \(\vect F:\mb R^n\to\mb R^n\) is conservative iff
        there exists
        a potential function \(f:\mb R^n\to\mb R\) such that
        \(\vect F=\nabla f\).
      \item (Example) Show that \(\vect W=\<2y+1,2x\>\) is conservative.
      \item (Example 7) Show that \(\vect V=\<y,-x\>\) is not conservative.
    \end{itemize}
  \item Flow Lines
    \begin{itemize}
      \item A flow line for a vector field \(\vect F:\mb R^n\to\mb R^n\)
        is a path \(\vect c:\mb R\to\mb R^n\) satisfying
        \(\vect c'(t)=\vect F(\vect c(t))\).
      \item (Example 8) Show that \(\vect c(t)=\<\cos t,\sin t\>\) is a flow
        line for \(\vect F=\<-y,x\>\), and find some other flow lines.
    \end{itemize}
  \item\textit{
    HW: 1-12, 17-21
  }
\end{itemize}

\section*{4.4 Divergence and Curl}
\begin{itemize}
  \item Divergence
    \begin{itemize}
      \item The divergence of a vector field
        \(\vect F:\mathbb R^n\to\mathbb R^n\) is denoted by
        \(\div{\vect F}:\mathbb R^n\to \mathbb R\) and defined by
        \(\div{\vect F} = \nabla \cdot \vect F = \sum_{i=1}^n \frac{\p F_i}{\p x_i}\)
      \item (Examples 3-5) Compute the divergences of \(\vect F=\<x,y\>\),
        \(\vect G=\<-x,-y\>\) and \(\vect H=\<-y,x\>\) at
        any point on \(\mathbb R^2\). How does divergence correspond with
        the motion described by the vector field plots?
      \item (Example) Compute the divergence of \(\vect F=\<x^2,y\>\) various
        points and interpret those values against a plot of the vector field.
    \end{itemize}
  \item Curl
    \begin{itemize}
      \item The curl of a three-dimensional vector field
        \(\vect F:\mathbb R^3\to\mathbb R^3\) is denoted by
        \(\curl{\vect F}:\mathbb R^3\to\mathbb R^3\) and defined by
        \(
          \curl{\vect F}
            =
          \nabla \times \vect F
            =
          \<
            \frac{\p F_3}{\p y}-\frac{\p F_2}{\p z},
            \frac{\p F_1}{\p z}-\frac{\p F_3}{\p x},
            \frac{\p F_2}{\p x}-\frac{\p F_1}{\p y}
          \>
        \)
      \item The scalar curl of a two-dimensional vector field
        \(\vect F:\mathbb R^2\to\mathbb R^2\) is denoted by
        \(\scurl{\vect F}:\mathbb R^2\to\mathbb R\) and defined by
        \(
          \scurl{\vect F}
            =
          \curl{\vect F} \cdot \veck
            =
          \frac{\p F_2}{\p x}-\frac{\p F_1}{\p y}
        \)
      \item (Example) Compute the scalar curl of \(\vect F=\<x,y\>\),
        \(\vect G=\<-x,-y\>\) and \(\vect H=\<-y,x\>\) at every point in
        \(\mathbb R^2\).
        How does this scalar curl correspond with
        the motion described by the vector field plots?
      \item (Example) Compute the curl of \(\vect F=\<y,-x,z\>\) at every
        point in \(\mathbb R^3\). How does curl correspond with the motion
        described by the vector field plot?
    \end{itemize}
  \item Facts about \(\nabla f\), \(\div \vect F\), \(\curl \vect F\)
    \begin{itemize}
      \item The curl of a conservative field is zero:
        \(\curl \nabla f = \nabla \times (\nabla f) = \vect 0\).
      \item (Example) Prove the above theorem.
      \item (Example) Prove that \(\vect F=\<x^2+z,y-z,z^3+3xy\>\) is not a
            conservative field.
      \item The divergence of a curl field is zero:
        \(\div \curl \vect F = \nabla\cdot(\nabla\times \vect F)=0\)
      \item Many identities on pg. 255 of Marsden text.
      \item (Example) Sketch proof of identity \#8:
        \(\div(\vect F\times\vect G)=\vect G\cdot\curl\vect F-\vect F\cdot\curl\vect G\).
    \end{itemize}
  \item\textit{
    HW: 1-4, 9-17, 22-25, 29-30
  }
\end{itemize}

\section*{5.3 The Double Integral Over More General Regions}
\begin{itemize}
  \item Hypervolume
    \begin{itemize}
      \item The hypervolume \(HV_1(D)\)
      of an interval \(D=[a,b]\) in \(\mathbb R\) is just its length \(b-a\).
      \item The hypervolume of a well-behaved bounded subset
      \(D\subseteq\mathbb R^{n+1}\) is defined for each
      \(n\in\{1,2,\dots\}\) by
        \[
          HV_{n+1}(D)
            =
          \int_{x_i\in I}
          HV(D_i)
          \dvar{x_i}
            =
          \int_{x_i=a}^{x_i=b}
          HV_n(D_i)
          \dvar{x_i}
        \]
      where \(I=[a,b]\) is an interval containing all values \(x_i\) included
      in the \(i\)th coordinate of \(D\), and \(D_i\) is the projection of
      of all points in \(D\) onto \(\mathbb R^n\) by removing the \(i\)th
      coordinate.
      \item (Example) For \(n=1\) and
      \(D=\{(x,y)\in\mathbb R^2:a\leq x\leq b,f(x)\leq y\leq g(x)\}\), we
      have that
        \[
          HV_2
            =
          A
            =
          \int_{x\in[a,b]}
          g(x)-f(x)
          \dvar{x}
            =
          \int_a^b
          g(x)-f(x)
          \dvar{x}
        .\]
      \item (Example) For \(n=2\) and \(D\subseteq R^3\) including values
      of \(x\) between \(a\) and \(b\), we have that
        \[
          HV_3
            =
          V
            =
          \int_{x=a}^{x=b}
          A(x)
          \dvar{x}
        \]
      where \(A(x)\) is the area of the cross-section of \(D\) taken by fixing
      each value of \(x\) (or similar for \(y\)).
    \end{itemize}
  \item Double Integrals
    \begin{itemize}
      \item
      For a bounded region \(D\subseteq\mathbb R^2\) and continuous
      nonnegative
      \(f:D\to\mathbb R\), the double integral
        \[
          \iint_D f\dvar A
        \]
      is defined to be the volume of
      \(\{(x,y,z)\in\mathbb R^3:(x,y)\in D,0\leq z\leq f(x,y)\}\).
      \item We may apply the definition of volume
      above to get
        \[
          \iint_D F\dvar A
            =
          \int_{x=a}^{x=b}
          A(x)
          \dvar x
        \]
      where \(D\) lies between the lines \(x=a\) and \(x=b\).
      \item If \(D\) is described by \(a\leq x\leq b\) and
        \(\phi_1(x)\leq y\leq \phi_2(x)\), then
        \[
          \iint_D F\dvar A
            =
          \int_{x=a}^{x=b}
          A(x)
          \dvar x
            =
          \int_{x=a}^{x=b}
          \left[
          \int_{y=\phi_1(x)}^{y=\phi_2(x)}
          f(x,y) \dvar y
          \right] \dvar x
        \]
      \item Similarly, if \(D\) is described by \(c\leq y\leq d\) and
        \(\psi_1(y)\leq x\leq \psi_2(y)\), then
        \[
          \iint_D F\dvar A
            =
          \int_{y=c}^{y=d}
          \left[
          \int_{x=\psi_1(y)}^{x=\psi_2(y)}
          f(x,y) \dvar x
          \right] \dvar y
        \]
      \item If \(f\) is sometimes negative on the domain \(D\), then
        \(\iint_D f\dvar A\) is the net volume between \(z=f(x,y)\)
        and \(D\) (volume above the
        \(xy\) plane minus volume below) and the above formulas still hold.
    \end{itemize}
  \item Iterated integrals
    \begin{itemize}
      \item An iterated integral is a shorthand for the expansion of two
        or more nested integrals, that is:
        \[
          \int_a^b
          \int_{\phi_1(x)}^{\phi_2(x)}
          f(x,y)
          \dvar y\dvar x
            =
          \int_{x=a}^{x=b}
          \left[
          \int_{y=\phi_1(x)}^{y=\phi_2(x)}
          f(x,y) \dvar y
          \right] \dvar x
        \]
      \item (Example) Sketch the region of integration for
            \(\int_0^\pi\int_{-x}^x \cos(y)\dvar y\dvar x\),
            evaluate it,
            and interpret it as the signed volume of a region in \(\mathbb R^3\).
      \item (Example) Express \(\iint_R (12x^3y-1)\dvar A\) where
            \(R\) is the
            rectangle with vertices \((0,0),(3,0),(3,2),(0,2)\)
            as an interated integral, then evaluate it.
      \item (Example) Express \(\iint_T (12x^3y-1)\dvar A\) where
            \(T\) is the
            triangle with vertices \((0,0),(1,0),(1,1)\)
            as an interated integral, then evaluate it.
    \end{itemize}
  \item Applications
    \begin{itemize}
      \item \(\iint_D 1\dvar A\) is the area of \(D\)
      \item \(\frac{1}{A(D)}\iint_D f(x,y)\dvar A\) is the average value
            of the function \(f\) restricted to \(D\)
    \end{itemize}
  \item Additivity
    \begin{itemize}
      \item
        If \(D\subseteq\mb R^2\) is the union of two subregions
        \(D_1,D_2\) overlapping only on their boundary, then
        \(\iint_D f\dvar V=\iint_{D_1}f\dvar V+\iint_{D_2}f\dvar V\).
      \item
        (Example) Prove that the area of the square with vertices
        \((1,0)\), \((0,1)\), \((-1,0)\), and \((0,-1)\) is two
        by setting it up as a double integral, then using additivity
        to split it up into two or more subregions.
    \end{itemize}
  \item\textit{
    HW: 1-9
  }
\end{itemize}

\section*{5.4 Changing the Order of Integration}
\begin{itemize}
  \item Rectangular regions of integration
    \begin{itemize}
      \item For constant bounds of integration:
        \[
          \int_a^b\int_c^d f(x,y)\dvar y\dvar x
            =
          \int_c^d\int_a^b f(x,y)\dvar x\dvar y
        \]
      \item (Example) Verify that
        \(
          \int_0^1\int_1^2 x^2+2xy\dvar y\dvar x
            =
          \int_1^2\int_0^1 x^2+2xy\dvar x\dvar y
        \).
    \end{itemize}
  \item Nonrectangular regions of integration
    \begin{itemize}
      \item Bounds of integration cannot be directly swapped; however, by
      interpreting the region of integration new bounds may be found in
      the other order.
      \item (Example) Verify that
        \(
          \int_0^4\int_0^{\frac{4-y}{2}} x+y \dvar x\dvar y
        \)
      and
        \(
          \int_0^2\int_0^{4-2x} x+y \dvar y\dvar x
        \)
      share the same region of integration and are equal.
      \item (Example) Evaluate
        \(
          \int_1^e\int_0^{\log x}\frac{(2x-e)\sqrt{1+e^y}}{e-e^y}\dvar y\dvar x
        \). %technically improper, if anyone asks
    \end{itemize}
  \item Estimating double integrals
    \begin{itemize}
      \item If \(g(x,y)\leq f(x,y)\leq h(x,y)\) for \((x,y)\in D\), then
      \(\iint_D g(x,y)\dvar A\leq\iint_D f(x,y)\dvar A\leq \iint_D h(x,y)\dvar A\).
      \item (Example 3) Prove that
      \(\frac{1}{\sqrt 3}\leq \iint_D \frac{1}{\sqrt{1+x^6+y^8}}\dvar A \leq 1\)
      where \(D\) is the unit square.
      \item (Example) Prove that
      \(e\leq\iint_D e^{x^2y+y}\dvar A\leq \frac{e^2}{2}\)
      where \(D\) is the unit square.
    \end{itemize}
  \item\textit{
    HW: 1-5, 7-10
  }
\end{itemize}

\section*{5.5 The Triple Integral}

\begin{itemize}
  \item Triple Integrals
    \begin{itemize}
      \item
      For a bounded region \(D\subseteq\mathbb R^3\) and nonnegative
      \(f:D\to\mathbb R\), the triple integral
        \[
          \iiint_D f\dvar V
        \]
      is defined to be the hypervolume of
      \(\{(x,y,z,w)\in\mathbb R^4:(x,y,z)\in D,0\leq w\leq f(x,y,z)\}\).
    \end{itemize}
  \item Applications
    \begin{itemize}
      \item \(\iiint_D 1\dvar V\) is the volume of \(D\)
      \item \(\frac{1}{V(D)}\iiint_D f(x,y,z)\dvar V\) is the average value
            of the function \(f\) restricted to \(D\)
      \item If \(\rho(x,y,z)\) gives the density of a solid at the coordinate
            \((x,y,z)\), then \(\iiint_D \rho(x,y,z)\dvar V\) calculates its
            overall mass.
    \end{itemize}
  \item Rectangular Boxes
    \begin{itemize}
      \item
      If \(B=[a_1,b_1]\times[a_2,b_2]\times[a_3,b_3]\), then
        \[
          \begin{matrix}
            \iiint_B f\dvar V
              & = &
            \int_{a_3}^{b_3}\int_{a_2}^{b_2}\int_{a_1}^{b_1}
            f(x,y,z) \dvar x\dvar y\dvar z
              \\ & = &
            \int_{a_2}^{b_2}\int_{a_1}^{b_1}\int_{a_3}^{b_3}
            f(x,y,z) \dvar z\dvar x\dvar y
              \\ & = &
            \text{etc.}
          \end{matrix}
        \]
      \item
      (Example)
      Write \(\iiint_D e^{x+y+z}\dvar V\) where
      \(D=[0,4]\times[0,2]\times[1,3]\) as a few different iterated integrals,
      then evaluate one.
    \end{itemize}
  \item General regions of integration
    \begin{itemize}
      \item
      If \(E\subseteq\mathbb R^2\) and
      \(D=\{(x,y,z)\in\mathbb R^3:(x,y)\in E,\gamma_1(x,y)\leq z\leq\gamma_2(x,y)\}\),
      then
      \[
        \iiint_D f(x,y,z)\dvar V
          =
        \iint_E
          \left[\int_{\gamma_1(x,y)}^{\gamma_2(x,y)} f(x,y,z)\dvar z\right]
        \dvar A
      \]
      (and similar for \(x,y\) instead of \(z\)).
      \item (Example 5) Express \(\iiint_W x\dvar V\) where \(W\) is the
      solid for which \(x,y,z\) are positive and \(x^2+y^2\leq z\leq 2\)
      as a few different iterated integrals.
      \item (Example 6) Express \(\iiint_W x\dvar V\) where \(W\) is the
      solid in \(\mathbb R^3\) above the triangle with vertices
      \((0,0,0),(1,0,0),(1,1,0)\) in the \(xy\) plane,
      and also between the surfaces \(z=x^2+y^2\)
      and \(z=2\), as an iterated integral. Then evaluate it.
    \end{itemize}
  \item Additivity
    \begin{itemize}
      \item
        If \(D\subseteq\mb R^3\) is the union of two subregions
        \(D_1,D_2\) overlapping only on their boundary, then
        \(\iiint_D f\dvar V=\iiint_{D_1}f\dvar V+\iiint_{D_2}f\dvar V\).
    \end{itemize}
  \item\textit{
    HW: 1-6, 11-17, 25-28
  }
\end{itemize}



\section*{1.4 Cylindrical and Spherical Coordinates}

\begin{itemize}
  \item Transformation of variables
    \begin{itemize}
      \item A transformation of variables is a function
            \(\vect T:\mathbb R^n\to\mathbb R^n\).
      \item (Example) Sketch the integer lattice on the \(uv\) plane and
            its image in the \(xy\) plane for
            the transformation of variables \(\vect T(u,v)=(x,y)=(u,u+v)\).
    \end{itemize}
  \item Polar Coordinates
    \begin{itemize}
      \item \(\vect p(r,\theta)=(r\cos\theta,r\sin\theta)\)
      \item \(r^2=x^2+y^2\), \(\tan\theta=\frac{y}{x}\)
      \item (Example)
            Convert \(A=\vect p(4,2\pi/3)\) from polar to Cartesian.
            Convert \(B=(3,-3)\) from Cartesian to polar. Plot both in
            the \(r\theta\) and \(xy\) planes.
      \item (Example)
            Express the curves \(x=\sqrt{4-y^2}\) and \(y=3\) in terms
            of polar coordinates. Plot both in the \(r\theta\) and \(xy\)
            planes.
    \end{itemize}
  \item Cylindrical Coordinates
    \begin{itemize}
      \item \(\vect c(r,\theta,z)=(r\cos\theta,r\sin\theta,z)\)
      \item Usually, assume \(r\geq0\) and \(0\leq\theta\leq2\pi\)
      \item \(r^2=x^2+y^2\), \(\tan\theta=\frac{y}{x}\)
      \item (Example 1) Convert \(A=\vect c(8,2\pi/3,-3)\) from cylindrical to
            Cartesian. Convert \(B=(6,6,8)\) from Cartesian to cylindrical.
            Plot both in \(xyz\) space.
      \item (Example) Express the surfaces \(x^2+y^2=9\) and \(z^2=x^2+y^2\)
            in terms of cylindrical coordinates. Plot both in \(xyz\) space.
    \end{itemize}
  \item Spherical Coordinates
    \begin{itemize}
      \item \(
              \vect s(\rho,\theta,\phi)
                =
              (
                \rho\sin\phi\cos\theta,
                \rho\sin\phi\sin\theta,
                \rho\cos\phi
              )
            \)
      \item Usually, assume \(\rho\geq0\), \(0\leq\theta\leq2\pi\), and
            \(0\leq\phi\leq\pi\)
      \item \(\rho^2=x^2+y^2+z^2\),
            \(\tan\theta=\frac{y}{x}\),
            \(\tan\phi=\frac{r}{z}=\frac{\sqrt{x^2+y^2}}{z}\)
      \item (Example 2) Convert \(A=(1,-1,1)\) from Cartesian to spherical.
            Convert \(B=\vect s(3,\pi/6,\pi/4)\) from spherical to Cartesian.
            Convert \(C=(2,-3,6)\) from Cartesian to spherical.
            Convert \(D=\vect s(1,-\pi/2,\pi/4)\) from spherical to Cartesian.
            Plot all four in \(xyz\) space.
      \item (Example 3) Express the surfaces \(xz=1\) and \(x^2+y^2-z^2=1\)
            in terms of spherical coordiantes.
    \end{itemize}
  \item\textit{
    HW: 1-11, 15
  }
\end{itemize}


\section*{6.1 The Geometry of Maps from \(\mb R^n\) to \(\mb R^n\)}
\begin{itemize}
  \item Images of regions by transformations
    \begin{itemize}
      \item (Example 1) Find the image of the rectangle
            \([0,1]\times[0,2\pi]\) in the \(r\theta\) plane under the
            polar coordinate transformation \(\vect p\).
      \item (Example 2) Find the image of the square
            \([-1,1]^2=[-1,1]\times[-1,1]\) in the \(uv\) plane under the
            transformation
            \(
              \vect T(u,v)
                =
              \begin{bmatrix}
                \frac{1}{2} & \frac{1}{2} \\
                \frac{1}{2} & -\frac{1}{2}
              \end{bmatrix}
              \<u,v\>
            \)
    \end{itemize}
  \item One-to-one and Onto
    \begin{itemize}
      \item A one-to-one transformation sends each point in the domain to
            a distinct point in the range.
      \item An onto transformation sends something in the domain onto each
            point of the range.
      \item (Example 3) Show that the polar coordinate transformation
            \(\vect p\) is onto but not one-to-one.
      \item (Example 4) Show that the transformation \(\vect T\) from example 2
            is both one-to-one and onto.
      \item (Example 5) Show that \(\vect T(u,v)=(u,0)\) is neither one-to-one
            nor onto.
      \item (Example 7) Find a rectangle in the \(r\theta\) plane which maps
            onto the region \(\{(x,y):x,y\geq 0,a^2\leq x^2+y^2\leq b^2\}\)
            in the Cartesian plane by the polar coordinate transformation.
    \end{itemize}
  \item Linear transformations
    \begin{itemize}
      \item Transformations \(\vect T:\mathbb R^n\to\mathbb R^n\) defined by
            \(\vect T(\vect u)=A\vect u\) for an \(n\)-dimensional matrix \(A\)
            are called linear transformations.
      \item (Example 6) Find a region in the \(uv\) plane which maps onto
            the square with vertices \((1,0),(0,1),(-1,0),(0,-1)\) in the
            \(xy\) plane by the linear transformation given in Example 2.
      \item Transformations \(\vect T:\mathbb R^n\to\mathbb R^n\) defined by
            \(\vect T(\vect u)=A\vect u+\vect x_0\) for an \(n\)-dimensional
            matrix \(A\) and \(n\)-dimensional vector \(\vect x_0\)
            are called affine transformations. (Every linear transformation
            is affine.)
      \item (Example) Find an affine transformation which maps the
            unit square in the \(uv\) plane onto the square with vertices
            \((1,0),(0,1),(-1,0),(0,-1)\) in the
            \(xy\) plane.
      \item An affine transformation is both one-to-one and onto
            exactly when \(\det A\not=0\).
      \item (Example) Use this fact to reinvestigate examples 4 and 5.
    \end{itemize}
  \item\textit{
    HW: 1-4, 8, 10
  }
\end{itemize}



\section*{6.2 The Change of Variables Theorem}

\begin{itemize}
  \item Affine transformations of areas
    \begin{itemize}
      \item An affine transformation with matrix \(M\) transforms hypervolumes
            by a factor of \(|\det M|\).
      \item (Example) Verify this fact for the parallelogram
            with vertices \((2,0),(3,1),(1,3),(0,2)\) in the \(uv\) plane
            and its image in the \(xy\) plane under
            the transformation \(\vect T(u,v)=(2u+v+3,v-u-2)\).
      \item Put another way, \(\iint_{D}1\dvar A=\iint_{D^*}|\det M|\dvar A\).
    \end{itemize}
  \item Affine transformations of single/double/triple integrals
    \begin{itemize}
      \item (Example) Let \(x=T(u)=mu+x_0\).
            Use substitution to prove that
            if the image of \([c_1,c_2]\) under \(T\) is \([b_1,b_2]\), then
            \(\int_{b_1}^{b_2}f(x)\dvar x=\int_{c_1}^{c_2}f(T(u))|m|\dvar u\).
      \item (Example) Use the previous fact to show that
            \(\int_0^4 \sqrt{2x+1}\dvar x=\int_1^9 \frac{1}{2}\sqrt{u}\dvar u\)
      \item For any 2D affine transformation \(\vect T\) with matrix \(M\)
            transforming \(D^*\) to \(D\),
            \(\iint_D f(x,y)\dvar A=\iint_{D^*}f(\vect T(u,v))|\det M|\dvar A\).
      \item (Example) Use an affine transformation to prove that
            \(
              \int_0^2\int_{y/2}^{(y+4)/2}2y\dvar x\dvar y
                =
              \int_0^1\int_0^1 16v\dvar v\dvar u
            \)
            and compute both integrals directly to verify.
      \item (Example)
            Compute \(\iint_D (x+y)(x-y-2)\dvar{A}\) where \(T\) is the
            triangle with vertices \((4,2)\), \((3,1)\), \((2,2)\).
      \item For any 3D affine transformation \(\vect T\) with matrix \(M\)
            transforming \(D^*\) to \(D\),
            \(\iint_D f(x,y,z)\dvar V=\iint_{D^*}f(\vect T(u,v,w))|\det M|\dvar V\).
    \end{itemize}
  \item Jacobian
    \begin{itemize}
      \item The Jacobian \(\frac{\p \vect T}{\p \vect u}\) of a transformation
            is defined to be the determinant of its partial derivative matrix:
            \(\det(\vect D\vect T)\).
      \item (Example) Prove that for an affine transformation \(\vect T\) with
            matrix \(M\) that \(\vect D\vect T=M\) and therefore
            \(\frac{\p\vect T}{\p\vect u}=\det M\).
      \item For any 2D transformation \(\vect T\)
            transforming \(D^*\) to \(D\),
            \(
              \iint_D f(\vect x)\dvar A
                =
              \iint_{D^*}f(\vect T(\vect u))|\frac{\p\vect T}{\p\vect u}|\dvar A
            \).
      \item For any 3D transformation \(\vect T\)
            transforming \(D^*\) to \(D\),
            \(
              \iiint_D f(\vect x)\dvar V
                =
              \iiint_{D^*}f(\vect T(\vect u))|\frac{\p\vect T}{\p\vect u}|\dvar V
            \).
      \item (Example) Use a 2D transformation to compute
            \(\iint_D e^x\cos(\pi e^x)\dvar{A}\) where
            \(D\) is the region
            bounded by \(y=0\), \(y=e^x-2\), \(y=\frac{e^x-1}{2}\).
    \end{itemize}
  \item Polar, cylindrical, spherical change of variables
    \begin{itemize}
      \item Polar coordinates:
            \(
              \iint_D f(x,y)\dvar A
                =
              \iint_{D^*}f(r\cos\theta,r\sin\theta)r\dvar A
            \)
      \item Cylindrical coordinates:
            \(
              \iint_D f(x,y,z)\dvar V
                =
              \iint_{D^*}f(r\cos\theta,r\sin\theta,z)r\dvar V
            \)
      \item Spherical coordinates:
            \(
              \iint_D f(x,y,z)\dvar V
                =
              \iint_{D^*} f(
                \rho\sin\phi\cos\theta,
                \rho\sin\phi\sin\theta,
                \rho\cos\phi
              )\rho^2\sin\phi\dvar V
            \)
      \item (Example 4) Evaluate \(\iint_D\log(x^2+y^2)\dvar A\) where \(D\)
            is the region in the first quadrant between the circles
            \(x^2+y^2=a^2\) and \(x^2+y^2=b^2\) for \(0<a<b\).
      \item (Example 6) Evaluate \(\iiint_W\exp[(x^2+y^2+z^3)^{3/2}]\dvar V\)
            where \(W\) is unit ball centered at the origin.
      \item (Example) Prove that the formula for the volume of a cone with
            radius \(R\) and height \(H\) is \(V=\frac{1}{3}\pi R^2H\).
      \item (Example 7) Prove that the formula for the volume of a sphere with
            radius \(R\) is \(V=\frac{4}{3}\pi R^3\).
    \end{itemize}
  \item\textit{
    HW: 1-6, 11, 13-14, 21, 26
  }
\end{itemize}



\section*{7.1 The Path Integral}

\begin{itemize}
  \item Path Integral with respect to Arclength
    \begin{itemize}
      \item Recall that for a curve \(C\) defined by
            \(\vect r:\mathbb R\to\mathbb R^n\), the arclength function
            \(s:\mathbb R\to\mathbb R\) defined by
            \(s(t)=\int_0^t\|\vect r'(\tau)\|\dvar\tau\) gives the length
            of the curve from \(0\) to \(t\).
      \item (Example) Prove that \(C=\pi D\) gives the circumference of
            a circle with diameter \(D\).
      \item If \(f:\mathbb R^n\to\mathbb R\) is a function defined
            along the curve \(C\) defined by \(\vect r:[a,b]\to\mathbb R^n\),
            then
            \[
              \int_C f\dvar s
                =
              \int_a^b f(\vect r(t))\frac{ds}{dt}\dvar t
            \]
            where \(\frac{ds}{dt}=\|\frac{d\vect r}{dt}\|\).
            This represents the area of a ribbon with base \(C\) and
            height \(f\) at each point of \(C\).
      \item (Example 1) Find the average value of the function
            \(f(x,y,z)=x^2+y^2+z^2\) along the portion of the helix given by
            \(\vect c(t)=\<\cos t,\sin t,t\>\) for \(t\in[0,2\pi]\).
      \item (Example 2) The base of a fence is given by the curve
            \(\vect c(t)=\<30\cos^3t,30\sin^3t\>\), and the height of the
            fence is given by \(f(x,y)=1+\frac{y}{3}\). How much paint is
            required to cover both sides of this fence?
    \end{itemize}
  \item\textit{
    HW: 1-8, 10-13
  }
\end{itemize}



\section*{7.2 Line Integrals}

\begin{itemize}
  \item Line Integral with respect to a Curve
    \begin{itemize}
      \item If \(\vect F:\mathbb R^n\to\mathbb R^n\) is a vector field defined
            along the curve \(C\) defined by \(\vect r:[a,b]\to\mathbb R^n\),
            then
            \[
              \int_C \vect F\cdot\dvar\vect r
                =
              \int_C \vect F\cdot\vect T\dvar s
                =
              \int_a^b \vect F(\vect c(t))\frac{d\vect r}{dt}\dvar t
            \]
            represents the work done by a force \(\vect F\) over the curve
            \(C\).
      \item (Example) An object is pushed around the unit circle with
            a force \(\<-y,x\>\) at each point \((x,y)\). Compute the
            work done in pushing the box around \(3\) full counter-clockwise
            rotations.
      \item (Example 1) Let \(\vect r(t)=\<\sin t,\cos t,t\>\)
            for \(t\in[0,2\pi]\)
            define the curve \(C\), and define the vector field
            \(\vect F=\<x,y,z\>\). Compute
            \(\int_C\vect F\cdot\dvar\vect r\).
      \item (Example 5) Let \(C\) be a circle in the \(yz\) plane centered
            at the origin.
            Show that no work is done by a force \(\vect F=\<x^3,y,z\>\)
            acting on an object moving around the circle.
    \end{itemize}
  \item Line integrals with respect to variables
    \begin{itemize}
      \item Note that for \(\vect r=\<x_1,\dots,x_n\>\),
            \[
              \int_C \vect F\cdot\dvar\vect r
                =
              \int_C \<F_1,\dots,F_n\>\cdot\<dx_1,\dots,dx_n\>
                =
              \sum_{i=1}^n \int_C F_i\cdot\dvar x_i
            \]
      \item If \(f:\mathbb R^n\to\mathbb R\) is a function defined
            along the curve \(C\) defined by \(\vect r:[a,b]\to\mathbb R^n\),
            then for \(1\leq i\leq n\)
            \[
              \int_C f\dvar x_i
                =
              \int_a^b f(\vect c(t))\frac{dx_i}{dt}\dvar t
            \]
      \item (Example)
            Compute \(\int_C xy\dvar y\) where \(C\) is the parabola defined
            by \(\vect c(t)=\<t,t^2,1\>\) for \(t\in[0,1]\).
      \item (Example 2) Evaluate and interpret
            \(\int_C x^2\dvar x+xy\dvar y+\dvar z\) where \(C\) is the
            parabola defined
            by \(\vect c(t)=\<t,t^2,1\>\) for \(t\in[0,1]\).
    \end{itemize}
  \item Reparametrizations and partitions
    \begin{itemize}
      \item The value of \(\int_C f\dvar s\)
            is independent of the choice of parametrization \(\vect r(t)\)
            regardless of orientation.
      \item The value of \(\int_C \vect F\cdot\dvar \vect r\)
            is independent of the choice of parametrization \(\vect r(t)\)
            provided it respects the orientation of \(C\).
      \item If \(C\) and \(-C\) represent the same curve with opposite
            orientations, then
            \(\int_C \vect F\cdot\dvar\vect r=-\int_{-C}\vect F\cdot\dvar\vect r\).
      \item If \(C=C_1+C_2\), then
            \(\int_C f\dvar s=\int_{C_1}f\dvar s+\int_{C_2}f\dvar s\) and
            \(
              \int_C \vect F\cdot\dvar \vect r
                =
              \int_{C_1}\vect F\cdot\dvar \vect r+\int_{C_2}\vect F\cdot\dvar \vect r
            \).
      \item (Example 11) Compute \(\int_C x^2\dvar x+xy\dvar y\) where
            \(C\) is the perimeter of the unit square oriented counter-clockwise.
    \end{itemize}
  \item\textit{
    HW: 1-5, 13, 17-18
  }
\end{itemize}



% \section*{8.1 Green's Theorem}

% \begin{itemize}
%   \item Green's Theorem
%     \begin{itemize}
%       \item Let \(\p D\) be the c.c.w. oriented boundary of a simple region
%         \(D\subseteq \mb R^2\). Then
%         \(
%           \int_{\p D}\vect F\cdot\dvar\vect r
%             =
%           \iint_D \scurl \vect F \dvar A
%             =
%           \iint_D \curl \vect F\cdot \veck \dvar A
%             =
%           \iint_D \frac{\p F_2}{\p x}-\frac{\p F_1}{\p y}\dvar A
%         \).
%       \item Note that the book lets \(\vect F=\<F_1,F_2\>=\<P,Q\>\).
%       \item (Example 1) Verify Green's Theorem for \(\vect F=\<x,xy\>\)
%         and \(D=\{(x,y):x^2+y^2\leq 1\}\).
%       \item (Example) Use Green's Theorem to prove that the area of \(D\)
%         is \(\frac{1}{2}\int_{\p D}x\dvar y-y\dvar x\).
%       \item (Example 3) Compute the work done using a force
%         \(\vect F=\<xy^2,y+x\>\) in moving an object from the origin to \((1,1)\)
%         along the curve \(y=x^2\) and then back to the origin along the line
%         \(y=x\).
%     \end{itemize}
%   \item\textit{
%     HW: 1-6, 9-10, 15
%   }
% \end{itemize}



% \section*{8.3 Conservative Fields}

% \begin{itemize}
%   \item Characterizations of Conservative Fields
%     \begin{itemize}
%       \item These are all equivalent to \(\vect F:\mb R^n\to\mb R^n\)
%         being conservative:
%         \begin{enumerate}[(1)]
%           \item There exists a potential function \(f:\mb R^n\to\mb R\) such
%             that \(\vect F=\nabla f\).
%           \item \(\curl \vect F=0\).
%           \item \(\vect F\) is path-independent: for any two curves \(C_1,C_2\)
%             which share starting and ending points,
%             \(
%               \int_{C_1}\vect F\cdot\dvar\vect r
%                 =
%               \int_{C_2}\vect F\cdot\dvar\vect r
%             \).
%           \item For any simple closed curve \(C\),
%             \(\int_C\vect F\cdot\dvar\vect r=0\).
%           \item For any curve starting at \(A\) and ending at \(B\), and any
%             potential function \(f\) for \(\vect F\):
%             \(\int_C\vect F\cdot\dvar\vect r=[f]_A^B=f(B)-f(A)\).
%         \end{enumerate}
%       \item (Example) Prove that (4) implies (3) above.
%       \item (7.2 Example 9) Evaluate \(\int_C y\dvar x+x\dvar y\) where
%         \(C\) is the curve given by \(\vect r(t)=(t^4/4,\sin^3(t\pi/2))\)
%         for \(t\in[0,1]\).
%       \item (Example 4) Find \(\int_C 2x\cos y\dvar x-x^2\sin y\dvar y\)
%         where \(C\) is given by \(\vect r:[1,2]\to\mb R^2\) defined by
%         \(x=e^{t-1},y=\sin(\pi/t)\).
%       \item (Example 1) Show that
%         \(\int_C \<y,z\cos yz+x,y\cos yz\>\cdot\dvar\vect r=0\) for any
%         simple closed curve \(C\).
%     \end{itemize}
%   \item\textit{
%     HW: 1-2, 5-8, 10-11
%   }
% \end{itemize}



% % \section*{7.3 Parametrized Surfaces}

% % \begin{itemize}
% %   \item Parametrization of a Surface
% %     \begin{itemize}
% %       \item Let \(S\subseteq \mathbb R^3\) be a surface and
% %             \(D\subseteq \mathbb R^2\)
% %             be a two-dimensional region. Then \(\vect\Phi:D\to S\) is a parametrization
% %             of \(S\) by \(D\).
% %       \item (Example) Show that the surface given by \(z=f(x,y)\)
% %             has the parametrization \(\vect\Phi(x,y)=\<x,y,f(x,y)\>\).
% %       \item (Example 1) Show that the plane passing through the point
% %             \(P\in\mathbb R^3\)
% %             and normal to the vector \(\vect a\times\vect b\) has a parametrization
% %             \(\vect\Phi(u,v)=\vect P+\vect au+\vect bv\).
% %       \item (Example 2) Show that the cone \(z=\sqrt{x^2+y^2}\) has a
% %             parametrization \(\vect\Phi(r,\theta)=\<r\cos\theta,r\sin\theta,r\>\)
% %             for \(r\geq 0,0\leq\theta\leq2\pi\).
% %       \item Surfaces which are conveniently described using cylindrical
% %             or spherical coordinates may be easily parameterized by adapting
% %             the relevant transformation.
% %       \item (Example) Use the cylindrical and spherical transformations to
% %             find parametrizations of the cone \(z=\sqrt{x^2+y^2}\).
% %     \end{itemize}
% %   \item Tangent and Normal Vectors to a Surface
% %     \begin{itemize}
% %       \item The tangent plane to a surface parameterized by \(\vect\Phi\)
% %             at the point \(\vect\Phi(\vect u_0)\) has parameterization
% %             \(
% %               \vect L(\vect u)
% %                 =
% %               \vect\Phi(\vect u_0)+[\vect D\vect \Phi(\vect u_0)]\vect u
% %             \).
% %       \item (Example 3) Find a parameterization of the plane tangent
% %             to the surface defined by
% %             \(\vect\Phi(u,v)=\<u\cos v,u\sin v,u^2+v^2\>\) at the point
% %             \((1,0,1)\).
% %       \item The columns \(\frac{\p\vect\Phi}{\p u}(u_0,v_0)\),
% %             \(\frac{\p\vect\Phi}{\p v}(u_0,v_0)\) give tangent vectors to
% %             the surface, and their cross-product is normal to the surface.
% %       \item (Example) Find an equation in \(x,y,z\) for the tangent plane
% %             in Example 3.
% %       \item (Example) Find a parameterization for the sphere cenetered at
% %             the origin with radius \(3\). Then describe the plane tangent
% %             to it at the point \((1,-2,2)\).
% %     \end{itemize}
% %   \item\textit{
% %     HW: 1-3, 7-11, 13-15
% %   }
% % \end{itemize}



% % \section*{7.4 Area of a Surface}

% % \begin{itemize}
% %   \item Definition of Surface Area
% %     \begin{itemize}
% %       \item The area of a surface parametrized by \(\vect\Phi\) with domain
% %             \(D\) is given by
% %             \(
% %               \iint_D
% %                 \|\frac{\p\vect\Phi}{\p u}\times\frac{\p\vect\Phi}{\p v}\|
% %               \dvar A
% %             \)
% %       \item (Example) Verify that this definition matches the area of the
% %             rectangle given by the vectors \(\<3,0,-4\>\) and \(\<0,-2,0\>\).
% %       \item (Example 1) Show that the surface area of a cone with slant length
% %             \(L\) and radius \(R\) is given by the formula
% %             \(A=\pi R^2+\pi RL\).
% %       \item (Example 2) Find that the area of a helicoid parameterized by
% %             \(\Phi(r,\theta)=\<r\cos\theta,r\sin\theta,\theta\>\) is
% %             equal to \(2\pi\int_0^1\sqrt{r^2+1}\dvar r\).
% %       \item TODO
% %     \end{itemize}
% %   \item\textit{
% %     HW: tba
% %   }
% % \end{itemize}

% \section*{Surface Integrals}

% \begin{itemize}
%   \item Definition
%     \begin{itemize}
%       \item If \(f:\mathbb R^n\to\mathbb R\) is a scalar function defined
%             on the surface \(S\) defined by \(\vect\Phi:\mathbb R^2\to\mathbb R^n\)
%             for \((u,v)\in D\), then
%             \[
%               \iint_S f(\vect x)\dvar S
%                 =
%               \iint_D
%                 f(\vect\Phi(u,v))
%                 \left\|\frac{\p\vect\Phi}{\p u}\times\frac{\p\vect\Phi}{\p v}\right\|
%               \dvar A
%             \]
%       \item (Example) Use \(A=\iint_S 1\dvar S\) to derive a formula for the
%             surface area of a cone in terms of its height \(H\) and angle
%             \(\theta\).
%       \item If \(\vect F:\mathbb R^n\to\mathbb R\) is a scalar function defined
%             on the surface \(S\) defined by \(\vect\Phi:\mathbb R^2\to\mathbb R^n\)
%             for \((u,v)\in D\) preserving orientation, then
%             \[
%               \iint_S \vect F(\vect x)\cdot\dvar\vect S
%                 =
%               \iint_D
%                 \vect F(\vect\Phi(u,v))
%                 \cdot
%                 \left(\frac{\p\vect\Phi}{\p u}\times\frac{\p\vect\Phi}{\p v}\right)
%               \dvar A
%             \]
%     \end{itemize}
%   \item Stokes' Theorem %TODO use uniform s,S in integrals
%     \begin{itemize}
%       \item If \(\p S\) is the positively oriented boundary of a surface
%         \(S\), then
%         \(
%           \iint_S\curl\vect F\cdot\dvar\vect S
%             =
%           \int_{\p S}\vect F\cdot\dvar\vect r
%         \).
%       \item (8.2 Example 2) Evaluate \(\iint_S(3x^2+3y^2)\veck\cdot\dvar\vect S\)
%         where \(S\) is the portion of the plane \(z=1-x-y\) above the unit
%         disk \(x^2+y^2\leq 1\) oriented toward positive values of \(x,y,z\).
%     \end{itemize}
%   \item Gauss'/Divergence Theorem
%     \begin{itemize}
%       \item If \(\p W\) is the outward oriented boundary of a solid
%         \(W\), then
%         \(
%           \iint_{\p W} \vect F\cdot\dvar\vect S
%             =
%           \iiint_{W} \div\vect F\dvar V
%         \).
%       \item (8.3 Example 3) Evaluate \(\iint_S\<2x,y^2,z^2\>\cdot\dvar\vect S\)
%         where \(S\) is the outward oriented boundary of the unit
%         sphere \(x^2+y^2+z^2=1\).
%     \end{itemize}
%   \item\textit{
%     HW: Review above examples
%   }
% \end{itemize}

% \section*{Overview of integration theorems}

% \begin{itemize}
%   \item \(\int_a^b f'(x)\dvar x=f(b)-f(a)=[f]_{\p[a,b]}\)
%   \item \(\int_C \nabla f\cdot\dvar \vect r=f(B)-f(A)=[f]_{\p C}\)
%   \item \(\iint_S \curl \vect F\cdot\dvar \vect S=\int_{\p S} \vect F\cdot\dvar\vect  r \)
%   \item \(\iiint_W \div \vect F\dvar V=\iint_{\p W} \vect F\cdot\dvar \vect S \)
% \end{itemize}



% % \section*{Remaining Topics}

% % \begin{itemize}
% %   \item 7.5 Integrals of Scalar Functions Over Surfaces
% %   \item 7.6 Surface Integrals of Vector Fields
% %   \item 8.2 Stokes' Thoerem
% %   \item 8.4 Gauss' Theorem
% % \end{itemize}

\end{document}